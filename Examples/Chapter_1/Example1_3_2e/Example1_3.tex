\documentclass[12pt,letterpaper]{article}
\usepackage[latin1]{inputenc}
\usepackage{amsmath}
\usepackage{amsfonts}
\usepackage{amssymb}
\usepackage{graphicx}
\input{../../../formatting3e}

\begin{document}
\verbatimfont{\small}

\section*{Example 1.3}
Consider the experimental observations of strength reported in Table \ref{Table:Ex:1:3} (p.~\pageref{Table:Ex:1:3}). Sort the
data into n=7 bins of equal width $\Delta S$ to cover the entire range of the data.  Build a histogram reporting  the number of observations per bin. Compute the mean $\mu_F$, standard deviation $\varpi_F$, and coefficient of variance $C_F$. Plot the normal probability density function $p(F)$ together with the normalized histogram. The histogram has to be normalized for plotting together with the probability density, which has area $A=1$. 

\subsection*{Solution}
The data $\bf F$ is sorted, then grouped into $n=7$ bins of equal width in Table \ref{Table:Ex:1:3} (Histogram).  Since the minimum strength is $F_{min}=30\;MPa$ and the maximum is $F_{max}=100\;MPa$, using 7 bins yield bins of with 10. 

The data is represented by a normal distribution with $\mu_F=65.4, \varpi_F=14.47, C_F=0.22$. The resulting histogram and PDF are shown in Figure \ref{Figure:1:Histogram}. 

%Hint: in MATLAB \verb+pdf('norm',x,m,v)+ evaluates the normal PDF $p(x)$ with mean \verb+m+ and standard deviation \verb+v+. Then, \verb+[Hist,Centers]=hist(F,7)+ returns the frequency and centers, and \verb+bar(Centers,Hist)+ plots the histogram.

\begin{verbatim}
% Example 1.3
clc; clear
[F txt] = xlsread('Table1_1.xlsx') % read data
ndata = length(F) % number of data points
% Histogram
n = 7; % number of bins
[Hist,centers] = hist(F,n)
binwidth = (max(F)-min(F))/n;
NHist = Hist/(ndata*binwidth); % normalizes the histogram 
bar(centers,NHist,'FaceColor','none','LineWidth',2); hold on
% probability density
F = sort(F); % sort for plotting a line
m = mean(F)
v = std(F)
p = pdf('norm',F,m,v)
plot(F,p,'k','LineWidth',2);
% labels
xlabel('Strength F','FontSize',14,'FontName','Arial'); 
ylabel('Probability p(F)','FontSize',14,'FontName','Arial');
axis([30 100 0.0 0.04]);
set(gca,'FontSize',14,'FontName','Arial'); 
legend('Histogram','PDF');
saveas(gca,'./Histogram.example.1.3','png');
hold off
\end{verbatim}
\end{document}